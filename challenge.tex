//
You have been given a random array of numbers, like:


(1) Create an algorithmic procedure to sort this based on the “find maximum” brute force solution. 

max_num = 0;
array_index=i;
array_lenght=15;
num[] = {1, 3, 5, 4, 5, 9, 6, 1, 2, 6, 5, 4, 4, 9, 7 };
count = 0;


for(i=0;i<array_lenght;i++){
 

if(max_num)
new_max_num=max_num;

 if(max_num==num[i]){
     num[i]=max_num;
 }else if (max_num<num[i]){
     num[i]=max_num;
 }else{
      num[i]=max_num;
 }
 

}
  count + 1;
  max_num = num;

else print max_num
